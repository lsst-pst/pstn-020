\documentclass[modern]{aastex62}

% lsstdoc documentation: https://lsst-texmf.lsst.io/lsstdoc.html
\input{meta}

% Package imports go here.

% Local commands go here.
\usepackage{tcolorbox}

\newcommand{\note}[1]{
   \begin{tcolorbox}[colback=red!5!white, colframe=red!75!black]
      #1
   \end{tcolorbox}
}

\newcommand{\docRef}{PSTN-020}
\newcommand{\docUpstreamLocation}{\url{https://github.com/lsst-pst/pstn-020}}


\begin{document}
\input{authors}
\date{\today}
\title{LSST Data Release Processing}
\hypersetup{pdftitle={\@title}, pdfauthor={\@author}, pdfkeywords={\@keywords}}

\input{abstract}


\section{Introduction}

   \note{
      Overall outline is strongly inspired by the HSC pipeline paper, though
      that will evolve as content is filled in.
      That paper relied heavily on having characterizations of pipeline performance
      in other papers, and I'm hoping we can do the same here.

      Ideally those would be DR1 papers, not construction papers, AND THIS WOULD BE
      TOO - publishing a DR pipeline paper before processing for the first DR has
      actually started is a recipe for wasting time writing throwaway text.
      We could substitute ``some informal DR based on commissioning data'' for ``DR1''
      above, if such a thing happens.  Important point is that this paper shouldn't
      go beyond outline form (except perhaps in the intro) until we start
      processing.
   }

\subsection{LSST/Rubin Overview}

\subsection{Terminology}

   \note{
      \begin{itemize}
         \item visits
         \item tracts
         \item patches
         \item footprints
         \item ...
      \end{itemize}
   }

\subsection{Related Papers}

   \note{
      \begin{itemize}
         \item Prompt products paper is obviously relevant.  Whether it comes out first depends on whether prompt processing needs to wait for DR1.  If it does come out first, we may want a new algorithms paper for both DR and prompt papers to refer to.  If not, might make more sense for prompt paper to refer to DR paper for algorithms.
         \item Will RHL's "Calibration Strategy" paper actually include the pipeline steps involved in making both traditional master calibrations, and fancier things like monochromatic flats and astrometric characterizations of the camera?  I am not including them below (yet), but they need to be covered somewhere.  Note that these are the kinds of things that have merited full papers of their own in other surveys.
         \item Need to identify papers that characterize pipeline performance (split into astrometry, photometry, ...?).  May come out later than this paper.  Trying to fully characterize pipeline performance while describing it is a recipe for a 200 page paper.
      \end{itemize}
   }

\section{Software Architecture}

   \note{
      \begin{itemize}
         \item This section is common to DR and prompt, and should only go in one of those papers.
      \end{itemize}
   }

\section{Pipelines and Data Products}

   \note{
      Organization here follows actual pipeline flow, and does not attempt to go into details.
      This is what we did in both the HSC Pipeline Paper and LDM-151.
   }

   \note{
      Present data products as a table, with references to the pipeline sections
      that finalize them.
   }

\subsection{Preliminary Image Characterization}

\subsection{Relative Photometric Calibration}

\subsection{Relative Astrometric Calibration}

\subsection{Final Image Characterization and Source Measurement}

   \note{Finalizes PVIs, Sources}

\subsection{Coaddition}

   \note{Finalizes coadd images}

\subsection{Image Differencing}

   \note{Finalizes difference images, DIASources}

\subsection{Object Definition and Measurement}

\note{Nearly-finalizes Objects, DIAObjects}

   \note{
      This will probably be multiple subsections later, but I don't yet know
      what they will be.
   }

\subsection{Forced Photometry}

\note{Finalizes ForcedSource}

\subsection{Postprocessing}

\note{Finalizes everything else.}

\section{Algorithms}

\subsection{Instrument Signature Removal}

\subsection{PSF Modeling}

\subsection{Aperture Corrections}

\subsection{Cosmic Ray Detection}

\subsection{Background Subtraction}

\subsection{Detection}

\subsection{Deblending}

\subsection{Source/Object Measurement}

   \note{
      Many subsubsections here.
   }

\section{Future Work and Conclusions}

   \note{
      What are the big known issues?

      What major improvements are planned?

      Reminder of related characterization papers.
   }


\appendix

% Include all the relevant bib files.
% https://lsst-texmf.lsst.io/lsstdoc.html#bibliographies
\section{References} \label{sec:bib}
\bibliographystyle{yahapj}
\bibliography{local,lsst,lsst-dm,refs_ads,refs,books}

% Make sure lsst-texmf/bin/generateAcronyms.py is in your path
\section{Acronyms} \label{sec:acronyms}
\input{acronyms.tex}

\end{document}
